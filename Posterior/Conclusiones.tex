\chapter*{Conclusiones}
Utilizando un monocromador SpectraPro 275 de la marca Action Research Corporation que se tenía en el laboratorio, un tubo foto multiplicador Hamamatsu, modelo H8249-101 y con un sistema desarrollado para el control de movimiento de un motor a pasos, se construyó un sistema espectrométrico automatizado y el mismo fue debidamente caracterizado.
El sistema es capaz de realizar mediciones dentro de un intervalo de los 250nm hasta los 700 nm, con un paso de 0.067nm que disminuye a 0.042nm. En este intervalo tiene un mejor poder de resolución que el de los espectrómetros HR4000 y QE65000 de la empresa Ocean Optics. \\

%La lampara de mercurio y la herramienta de \textbf{curve fitting tool}, fueron de gran utilidad para encontrar una relación paso a longitud de onda. \\
%
%La lampara de tungsteno-halogeno nos ayudo a ver el intervalo en el que se puede medir con el sistema.\\
%
%Los espectros medidos con este sistema servirán para identificar lineas de emisión pero al no tener una calibración en potencia, la forma del espectro no será la correcta, en ciertos intervalos del espectro. \\


