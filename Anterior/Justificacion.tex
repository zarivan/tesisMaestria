\specialhead{Justificación}
La línea de investigación de instrumentación fotónica de la SEPI-ESIME-Z cuenta con diversos miniespectrómetros para la realización de sus investigaciones. Estos miniespectrómetros permiten en principio la captura automática de espectros de luz continua  con resoluciones de hasta 1 nm. El automatizar un monocromador permitirá realizar, además de espectroscopia de luz continua, espectroscopia resuelta en tiempo y de mayor resolución 0.5 nm y más sensibilidad.


Al ser un sistema desarrollado en el propio laboratorio, este podrá ser modificado con facilidad con la finalidad de mejorar sus características o para aplicaciones particulares. Otro de los beneficios que se obtiene con automatizar el monocromador, es el bajo costo que esto implica en comparación con la compra de un sistema comercial automatizado. Con el sistema propuesto se podrá además:
\begin{itemize}
\item Estudiar fuentes luminosas de baja intensidad.
\item Podrá ser modificado para realizar mediciones en otros intervalos del espectro de luz con el cambio de la rejilla de difracción y el tubo fotomultiplicador.
\end{itemize}

%\paragraph{Aplicaciones} 
Una de las aplicaciones requeridas y bien definidas en el laboratorio es que, aprovechando que el monocromador tiene un alto poder de resolución, al combinarlo con un tubo fotomultiplicador de alta sensibilidad, será posible medir espectros de emisión de los plasmas estudiados en el grupo de instrumentación fotónica.
