\tableofcontents
\listoffigures
\listoftables
\specialhead{Resumen}
La espectroscopia es una técnica de medición para estudiar la interacción de la radiación electromagnética con la materia.
Dentro del amplio espectro electromagnético, el presente trabajo se enfocará en la luz ultravioleta (UV) y la luz visible (VIS). Hoy en día existen en el mercado equipos capaces de realizar espectroscopia óptica, sin embargo, la mayoría tienen costos elevados. Aprovechando equipo de laboratorio con el cual se cuenta, se ha desarrollado un sistema espectroscópico automatizado. \\
El sistema consta de un monocromador SpectraPro 275, un módulo de tubo fotomultiplicador (PMT) y un motor a pasos. Los cuáles serán controlados por un microcontrolador y una PC.\\
El intervalo de trabajo de este espectrómetro es de los 200 hasta los 650nm, con un paso promedio de 0.05nm. \\
La Interfaz gráfica fue diseñada en LabView, se utilizó un microcontrolador para el control y un ADC para la adquisición de los datos.
