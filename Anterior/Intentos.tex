\begin{table}
	\centering
	\caption{Tabla donde se muestran las diferentes combinaciones de los \textit{switchs}, para limitar la corriente de trabajo.}
	\label{table:corrienteTrabajo}
	\begin{tabular}{|c|c|c|c|c|c|c|c|c|c|c|c|c|c|c|}
		\hline 
		\multicolumn{15}{|c|}{CORRIENTE DE TRABAJO} \\ 
		\hline 
		& 0.3 & 0.5 & 0.8 & 1.0 & 1.1 & 1.2 & 1.4 & 1.5 & 1.6 & 1.9 & 2.0 & 2.2 & 2.6 & 3.0 \\ 
		\hline 
		SW1 & OFF & OFF & OFF & OFF & OFF & ON & OFF & ON & ON & ON & ON & ON & ON & ON \\ 
		\hline 
		SW2 & OFF & OFF & ON & ON & ON & OFF & ON & OFF & OFF & OFF & OFF & ON & ON & ON \\ 
		\hline 
		SW3 & ON & ON & OFF & OFF & ON & OFF & ON & ON & OFF & ON & ON & ON & OFF & ON \\ 
		\hline 
		S1 & ON & OFF & ON & OFF & ON & ON & OFF & ON & OFF & OFF & OFF & ON & OFF & OFF \\ 
		\hline 
	\end{tabular} 
\end{table}
\begin{table}
	\begin{tabular}{|c|c|}
		\hline 
		\multicolumn{2}{|c|}{Corriente Máxima} \\ 
		\hline 
		S2 &  \\ 
		\hline 
		ON & 20\% \\ 
		\hline 
		OFF & 50\% \\ 
		\hline 
	\end{tabular}
	
	\begin{tabular}{|c|c|c|}
		\hline 
		\multicolumn{3}{|c|}{Control de tipo de paso} \\ 
		\hline 
		& \multicolumn{2}{c|}{\textit{switch}} \\ 
		\hline 
		Paso & S3 & S4 \\ 
		\hline 
		Completo & OFF & OFF \\ 
		\hline 
		$\frac{1}{2}$ & ON & OFF \\ 
		\hline 
		$\frac{1}{8}$ & ON & ON \\ 
		\hline 
		$\frac{1}{16}$ & OFF & ON \\ 
		\hline 
	\end{tabular} 
\end{table}